\documentclass[10pt,a4paper]{article}
\usepackage[english]{babel}
\usepackage[utf8]{inputenc}
\usepackage{amsmath}
\usepackage{amsfonts}
\usepackage{amssymb}
\usepackage{hyperref}
\title{Research Proposal for the Introduction to Computational Science Project}
\author{Marc Went, student number 10905847\\Ferry Avis, student number 10904581}
\begin{document}

\maketitle

\section*{Introduction}

This document describes our approach for the modelling and simulation project for the course Introduction to Computational Science of the college year 2014-2015 at the University of Amsterdam.

The aim of the project is to perform a science project, which will be supported by a simulation study. The steps we take are as follows: we ask a question, propose, implement, perform and validate an experiment, and describe and analyse the results. This report focusses on the first two steps. 

\section*{Problem description}

In 1968, a group of people from politics, business and science came together to \emph{"discuss the dilemma of prevailing short-term thinking in international affairs and, in particular, the concerns regarding unlimited resource consumption in an increasingly interdependent world"}. This group was named \emph{The Club of Rome} [1], after the location of the meeting.

The first and subsequent meetings resulted in the book `\emph{Limits to Growth}', published in 1972. In the book it was predicted that economic growth could not continue indefinitely because of the limited availability of natural resources, particularly oil. In 1974 a second report was published, `\emph{Mankind at the Turning}'. This report was more optimistic and noted that many factors where within human control. Environmental catastrophes are preventable [2].

Today, the Club of Rome still exists. The views of the Club of Rome are still broadly correct: unlimited consumption and growth on a planet with limited resources cannot go on forever. The focus of the group does not longer lie on natural resources only. The group identifies the most crucial problems which will determine the future of humanity, evaluates alternative scenarios, proposes solutions and communicates the insights to decision-makers in the public and private sectors and general public [3].

In our research project, we will model the usage of the oil consumption in the world, see if growth (if any) can be continued and evaluate alternative scenarios for the future.

\section*{Approach}

For the mathematical model of the oil consumption, we will look at `Limits to Growth'. The model consists of five variables: world population, industrialization, pollution, food production and resource depletion [4].

We will perform a sensibility analysis to see how the model changes when parameters are varied. In this way, we can see what parameters are most influential and support decision makers in making decisions to avoid a shortage of oil. Additionally, we will try to design and implement other methods for modelling the growth. We will compare the models to study the differences.

The experiments will be carried out by computer simulations. We are planning to use Python as the tool for performing our experiments.

Unfortunately, we did not manage to see a copy the book of Shiflet and Shiflet, used in the course, yet. We will read the book to see if we can use it for inspiration on a model based on differential equations.

Additionally, we will perform a small literature study on the subject. We expect that there exists several models and solutions for the oil consumption problem.

\section*{Background of the team members}

Currently, we are both following the Computer Science master programme at the VU University in Amsterdam, with High Performance Distributed Computing as our specialisation. Marc has been doing the Computer Science bachelor, while Ferry has a background in Econometrics. We will combine our strengths, which lie in programming and mathematics respectively, to teach each other as much as possible. Ferry has experience with programming, but not in Python. He will be able to help Marc with mathematics, while Marc will help Ferry with Python.

\section*{References}

\begin{enumerate}
\item History page on The Club of Rome website, retrieved September 6th, 2014, \url{http://www.clubofrome.org/?p=375}
\item Club of Rome Wikipedia page, retrieved September 6th, 2014, \url{http://en.wikipedia.org/wiki/Club_of_Rome}
\item About page on The Club of Rome website, retrieved September 6th, 2014, \url{http://www.clubofrome.org/?p=324}
\item The Limits of Growth Wikipedia page, retrieved September 6th, 2014, \url{http://en.wikipedia.org/wiki/The_Limits_to_Growth}
\end{enumerate}
\end{document}
\documentclass[10pt,a4paper]{scrartcl}
\usepackage[utf8]{inputenc}
\usepackage[english]{babel}
\usepackage{color}

\title{Study on the Club of Rome's World3 Model}
\subtitle{Project Introduction to Computational Science 2014\\
University of Amsterdam}
\author{Marc Went, student number 10905847\\Ferry Avis, student number 10904581}
\usepackage{natbib}

\begin{document}

\maketitle

\section*{Introduction}

This is our report the modelling and simulation project for the course Introduction to Computational Science of the college year 2014-2015 at the University of Amsterdam.

The aim of the project is to perform a science project, which will be supported by a simulation study. 

\section*{Background of the model}

In 1968, a group of people from politics, business and science came together to \emph{"discuss the dilemma of prevailing short-term thinking in international affairs and, in particular, the concerns regarding unlimited resource consumption in an increasingly interdependent world"}. This group was named \emph{The Club of Rome}, after the location of the meeting. The first and subsequent meetings resulted in the book `\emph{Limits to Growth}', published in 1972. In the book it was predicted that economic growth could not continue indefinitely, mainly because of the limited availability of natural resources. At some point, the quality of life of humans would collapse. The time scale to take action was rather short: only 75 years. Today, the conclusions from the Club of Rome still have a huge impact.

The underlying model used in the study is the World3 model. It was the first model that used computer simulation to model the world. The model is the result of the interaction of of five sub models, which represent capital, resources, agriculture, population and pollution.

The World3 model is far from perfect and was merely a first step in modelling the world. The authors note the use of inadequate data, the usage of quantifications of factors for which the influence is not fully clear, such as pollution, and unknown development of technology in the future. Also, natural resources are seen as one identity, while substitutions are possible. Increasing prices because of scarcity and the stabilising working of the law of supply and demand are missing. The assumptions used in the model are very different from assumptions economists often make. However, the model was well enough to illustrate the correctness of the Club of Rome's hypotheses. The model is an example that it is not necessary to have a perfect model to give an clear insight in problems. The model does not give a prediction, but shows the general direction the world is heading to when no action is taken.

No single country is responsible for the problems on its own or able to take appropriate actions. Measures on a world wide scale are necessary, which are very difficult to coordinate. The history of the ozone hole and acid rain shows that global coordination is not impossible. However, current problems are far 

One of the examples of the use of incorrect data was the amount of available natural resources. New techniques have made it, for example, possible to extract oil deeper from the earth. However, as we will see, this will only delay the point \textcolor{red}{of collapse}.

The book has received many critics, mainly by people who did not understand the main message. As stated earlier, the used assumptions are very different economists often use. Also, the tone of the report is pessimistic. The Club of Rome was wrongfully branded as the zero growth movement. We need to strive for sustainable growth, instead of continuing the direction we are currently heading to. Poor countries are able to grow.




\section*{Standard run of the model}

Shows the direction when no actions are taken.

Missing scale: it is not a prediction

\section*{Hypotheses}

Influence of population

Illustration of the delay of collapse when natural resources are increased

\section*{Implementation of the model}

Python

Describe validation too

\section*{Results and analysis}



\section*{The Club of Rome today}

Today, the Club of Rome still exists. The views of the Club of Rome are still broadly correct: unlimited consumption and growth on a planet with limited resources cannot go on forever.

\section*{Conclusion}



\bibliographystyle{abbrvnat}
\nocite{*}
\bibliography{bibliografie}


\end{document}